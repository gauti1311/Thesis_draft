\section{Motivation}
In industrial automation, the market needs reliable and scalable solutions for autonomous transportation in production and logistic processes. To address this need, SICK AG e.g. offers reliable LiDaR localization solutions and a big portfolio of LiDaR sensors. However, the
increased usage of small mobile platforms in swarm applications introduces additional requirements compared to historically bigger autonomous vehicles: The cost factor regarding number and type of used sensor systems increases, the performance of the hardware is limited
and the environment changes.The main target of this work is the investigation, adaption and evaluation of visual Odometry
algorithms for application in mobile swarm robotics.

\section{Thesis Structure}

The thesis is structured into seven chapters. The Chapter 1 which serves as an introduction.

Chapter 2 covers the basics of Visual Odometry. It explains the different methods of VO and a general process of VO. Furthermore it describes current state-of-the-Art VO algorithms and selection criteria for chosen ones for further work.

Chapter 3 explains in depth the 3 chosen algorithms LDSO, SVO, ORB-SLAM2. Code flowcharts are explained.

Chapter 4 provides information regarding the Mobile robot and cameras used for the implementation and required parameters like Intrinsic and Photometric Camera Calibration.

Chapter 5 describes the procedure of experiments and data collection. All algorithms are then evaluated. The results are then discussed.

Chapter 6 gives suggestions how to choose the best algorithm and to improve results furthermore.

Chapter 7, concludes the thesis by summarizing the results and suggestions for future works.