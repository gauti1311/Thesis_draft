The goal of this thesis was software implementation state of the art \acrshort{vo} methods in addition to hardware integration on mobile robot for use case in warehouse scenario. The literature research was done first and three state of the art \acrshort{vo} methods based on different approaches were selected to implementation. They were \acrshort{orb}-\acrshort{slam}, \acrshort{ldso} and \acrshort{svo}. Total two cameras were also selected to compare the results and evaluate the properties that affect most to \acrshort{vo}. Selected \acrshort{vo} methods are explained in detailed including their working principles, modules and code implementation. An important task before integration to mobile robot is camera calibration taking care of lens distortion. Based on experience some good practices were given to perform accurate camera calibration which may be helpful in. A 3D pose estimation approach was selected to perform hardware integration and select \acrshort{lidar} as base frame which was supposed to used in evaluation with ground truth.\\
\newline
All three algorithms were implemented in software framework as shown in figure \ref{fig:implementation}.
Some parameters tweaking and improvements was done on all of them according to camera and environment to have best possible performance. Total 5 different types of experiments were taken to evaluate various evaluation matrices as shown in figure \ref{fig:matrices}. The results showed that \acrshort{orb}-\acrshort{slam} performed well in all data sets with highest robustness and accuracy followed by \acrshort{ldso}. \acrshort{ldso} performed poor in during rotation in camera motion. In terms of efficiency \acrshort{svo} shows very good results capable to real time applications as it was made for that purpose. \acrshort{orb}-\acrshort{slam} followed \acrshort{svo} in speed and \acrshort{cpu} usage.
An attempt was done to improve the scale estimation of \acrshort{orb}-\acrshort{slam} using ground plane constraint but it was not good enough due to inconsistency features on ground. The \acrshort{orb}-\acrshort{slam} and \acrshort{ldso} are then compared with \acrshort{lidar} odometry adding absolute scale factor. \\
\newline
Quantitative evaluation very important real-time \acrshort{vo} applications. Unfortunately, this thesis was not able to prepare any ground truth. By integrating extra sensor to estimate absolute scale of camera motion \acrshort{orb}-\acrshort{slam} can be quantitatively evaluated with \acrshort{lidar} odometry and ground truth which can be done in future. \acrshort{orb}-\acrshort{slam} has full potential for real time applications.\\
\newline
To summarize, this thesis showed working of state of the art \acrshort{vo}, their ability to perform in warehouse scenario and weaknesses. An in-depth explanation was given on selection of cameras and their integration. Furthermore it provided some improvements which can be done to improve the performance and to run independently on a mobile plateform in real-time.